
O capítulo de conclusões é um dos mais importantes do relatório, devendo ser apresentado um resumo dos resultados do trabalho efetivamente desenvolvido. As conclusões devem basear-se nos resultados realmente obtidos. Devem enquadrar-se os resultados obtidos com os objetivos enunciados e procurar extrair conclusões mais gerais, eventualmente sugeridas pelos resultados. Podem acompanhar as conclusões incluindo recomendações apropriadas resultantes do trabalho, nomeadamente sugerindo e justificando eventuais extensões e modificações futuras.

\section{Objectives achieved}
Nesta secção devem ser repetidos os objetivos apresentados no capítulo de introdução e para cada um deles deve ser descrito o seu grau de realização. Recomenda-se o uso de uma lista ou tabela, dado que facilita a leitura e compreensão.

\section{Limitations and future work}
Nesta secção devem ser identificadas as limitações do trabalho realizado, fazendo uma análise autocrítica do trabalho realizado, bem como extrapolar eventuais direções de desenvolvimento futuro.

\section{Final assessment}
Esta secção deve fornecer uma opinião pessoal sobre o trabalho desenvolvido.

