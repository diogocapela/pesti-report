Computer aided detection (CAD) is a clinically proven technology that increases the detection of clinical signs and diseases by assisting the doctors in decreasing observational oversights. The more recent clinical introduction of CAD to assist doctors in the detection of pulmonary signs and diseases will likely be followed by the development, clinical trials validation, regulatory approval and commercialization of a variety of CAD applications in diagnostic imaging. \cite{Castellino2005}
\\ \\
These kinds of software are being used in all fields of medicine but because of the nature of the data itself it is more used in areas where the clinical diagnosis is mostly made by using visual tools, like radiology for example. The computer analyses the image and symptomatology data and outputs a likelihood percentage to aid the doctor in his/her diagnosis.
\\ \\
Nonetheless, making a diagnosis based on images of a patient is often not an easy task. The idea of using computers to help, and possibly improve the interpretation of medical images is therefore very appealing. The first reference to the term computer-aided diagnosis that I have found is in a paper from 1963 by G.S. Lodwick, in Investigative Radiology 1(1):72-80 entitled "Computer-aided diagnosis in radiology. A research plan." \cite{Lodwick1966}


\section{Related works}

There are many companies investing in this field of aided diagnostics, but when analyzing it just for the ones who are focusing on tuberculosis, we can say that EPCON has two main competitors in the market:

\begin{itemize}

\item
\textbf{Qure.ai} from the United States, which has a product named \textbf{qXR} that detects abnormal chest X-rays, then identifies and localizes 15 common abnormalities and screens for tuberculosis. It is used in public health screening programs and was trained with over a million curated X-rays and radiology reports, making it hardware-agnostic and robust to variations in X-ray quality. \cite{QureAI}

\item 
\textbf{Delft Imaging} from the Netherlands, that developed \textbf{CAD4TB} in order to help (non-expert) readers detect tuberculosis more accurately and cost-effectively through the speed of digital X-rays combined with deep learning and remote expertise. \cite{DelftImaging}
	
\end{itemize}

\section{Existing Technologies}

Bla bla bla

