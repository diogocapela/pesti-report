% !TEX encoding = UTF-8 Unicode

\pagenumbering{arabic}

The proposed challenge by EPCON was to develop on top of an already existing system that uses AI technology to help doctors and other healthcare workers screen patients in risk of contracting TB. This project from EPCON was focused on computer-aided diagnosis, specifically for tuberculosis, and for under-developed world regions.

\section{Organization description}

EPCON is a small startup company based in Antwerp, Belgium that since 2018 focuses on identifying, monitoring and evaluating epidemic outbreak regions and finding those in need of help through technology and innovation.
\\ \\
Through their partner network, that consists of a vast list of international companies, NGOs and local communities that have proven track records and unique expertise in the field of GIS, AI and healthcare, they consolidate expertise in terms of health data and commit to further innovate and develop technologies to help epidemic control and fight against TB.
\\ \\
By doing this, they aggregate real-time data and combine it with contextual geographical information like population density, disease co-infection indexes, migration ratios and others. And then, using their distributed Bayesian reasoning models, they analyse cause-effect patterns and use neural networks to screen chest x-rays at large scale for computer-aided diagnostics. \cite{EPCON}

\section{Problem description}

One of the already built technologies of EPCON consists of a small web application that connects to their internal API allowing users to upload a chest x-ray photo and receiving a computer generated estimation on how likely is the x-ray to indicate that the patient could have tuberculosis.
\\ \\
After some testing with real doctors in developing countries, EPCON asked the users for some feedback about the usability of the application and they replied with some issues that concerned their use of it.
\\ \\
One of the biggest complaints was referent to the upload process of the photos of x-rays where the process was taking too long and when the connection was unstable it would fail the request and stop uploading.

\section{Approach}

Bla bla bla

\section{Contributions}

Bla bla bla

\section{Work planning}

We were 10 students from 7 different countries working together remotely for approximately 3 months. We have a first initial meeting during one week in Belgium where we had the chance to meet each other, to get to know the people behind the company and to be elucidated about more specific details on the project we were about to develop.
\\ \\
We decided to have a weekly meeting, every Thursday at 20:00 CET using an online video-conference platform provided to us by the company. Where we discussed all the topics and issues we had during the previous week and all the tasks and user stories we were about to develop on the next week. Vincent, the CEO of EPCON, would join this meeting intercalary, so every two weeks.
\\ \\
In order to communicate with each other we used Slack as it had a free-tier option and it allowed us to communicate, send files and make voice calls.
\\ \\
To manage our workflow and write down all the user stories and their respective status we decided to use Trello, a web-based Kanban-style list-making application. \cite{Trello} We first thought about using Jira, an agile project management software developed by Atlassian, but, as we had some team members who were not very experienced with scrum-based workflow and software development in general, we decided to go with Trello as it seemed a much more simpler, user-friendly option which could fulfill all our organizational needs.
\\ \\
In order to build software in a continuous development fashion we need a version control system where we could track version changes of our code repository. Nowadays, it is almost industry standard to use Git as a version control software, but there are other options like Mercurial or SVN, so we decided to use Git. As a Git platform we used GitLab, as it offered free private repositories and free CI/CD tools.

\section{Report structure}

Apresentação sucinta dos capítulos que fazem parte do relatório, descrevendo em poucos parágrafos o que cada um deles trata. 
Para além da introdução, esta dissertação contém mais x capítulos. No capítulo 2, é descrito o estado da arte e são apresentados trabalhos relacionados. No capítulo 3...




