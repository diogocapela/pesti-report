% !TEX encoding = UTF-8 Unicode

%----------tipo de documento------------------
\documentclass[openany,11pt,twoside,a4paper]{report}

%-------------- HEADER E FOOTER ------------
\usepackage{fancyhdr}
\usepackage[utf8]{inputenc}

\pagestyle{fancy}                       
\fancyheadoffset{0.1cm}
%---------Header--------
\fancyhead[LE,RO]{}			
\fancyhead[RE,LO]{\small{Screen missing patients using smart and innovative solutions to eradicate tuberculosis}}
%---------Footer---------
\fancyfoot{}
\fancyfoot[RE,LO]{\small{Diogo Capela}}				%
\fancyfoot[LE,RO]{\thepage}

%------------ First chapter page ----------------
\fancypagestyle{plain}{%
	\fancyhf{}%
	\fancyfoot{}
	\fancyfoot[RE,LO]{\small{Diogo Capela}}				
	\fancyfoot[LE,RO]{\thepage}
	\renewcommand{\headrulewidth}{.0pt}% Line at the header invisible
	\renewcommand{\footrulewidth}{0.5pt}% Line at the footer visible
}

%--------- Rules -------------------
\renewcommand{\headrulewidth}{0.5pt}    % Width of head rule 
\renewcommand{\footrulewidth}{0.5pt}     % Width of footer rule 

% ==================
% Configuração página
% ==================
\setlength\parindent{0pt}

% -----Topo------------------
\setlength{\topmargin}{-\headheight}
\setlength{\headsep}{0.43cm}

% -----Espaçamento---------------
\renewcommand{\baselinestretch}{1.5}   % espacamento de 1.5 linhas

%------ Tabelas -----
\usepackage{longtable}
\usepackage
[
        a4paper,% other options: a3paper, a5paper, etc
        left=3.1cm,
        right=3.1cm,
        top= 2.5 cm,
        bottom=2.5 cm,
]
{geometry}

% ------------------ Opções em tabelas-----------------
\usepackage{array}		% para centrar colunas (de tabelas) com p{XXcm}
\usepackage{paralist}	% para colocar enumerates em linha com um parágrafo

\usepackage{booktabs,multirow} 
\usepackage{longtable}
\usepackage{tabularx}
\usepackage{subcaption}

\usepackage{tabu}
\usepackage{tabulary}

\usepackage[section]{placeins}


%--------- Texto com cores --------------
\usepackage[pdftex]{color,graphicx} %letras


% -----------------PORTUGUES----------
\usepackage[utf8]{inputenc}
\usepackage[T1]{fontenc}

%-----tipo letra---------
\usepackage{helvet} % --- não existe Calibri em  pdflatex
\renewcommand{\familydefault}{\sfdefault}

%Portuguese-specific commands
\usepackage[english]{babel}
 
% -----------------   Hyphenation rules ----------------
\usepackage{hyphenat}
\hyphenation{re-cu-pe-rar te-ma a-re-a}

%-------------------------------------------------------------------------------------------------------
%para palavras que ao quebrar leva 2 hifen ,: por exemplo verificou-se: no texto fazer verificou{-}{-}se
\defineshorthand{"-}{\nobreak\hskip0pt\discretionary{-}{-}{-}\nobreak\hskip0pt} 
%-----------------------------------------------------------------------

% ---------- Hyperlinks ------------
\usepackage{cooltooltips} 
\def\cool{\texttt{cool}}

\usepackage{verbatim}

% ------------------ÍNDICE COM LINKS ------------------
\usepackage{hyperref}
\hypersetup{
    colorlinks=true,
    citecolor=black,
    filecolor=black,
    linkcolor=black,
    urlcolor=black,
    linktoc=all
}

%--------------------------------------------------------------------------------
%-----------------Para Titulos: capitulos, secções, etc.--------------------------
%--------------------------------------------------------------------------------
\usepackage{titlesec}

%----------Retirar prefixo capítulo e acrescentar numero seguido de ponto----
\titleformat{\chapter}[hang]{\bfseries\huge}{\thechapter.}{2pc}{} 
\titlespacing*{\chapter}{0pt}{0pt}{15pt}

%----------Tamanho da letra dos títulos das secções----
\titlespacing\section{0pt}{12pt plus 4pt minus 2pt}{0pt plus 2pt minus 2pt}
\titlespacing\subsection{0pt}{11pt plus 4pt minus 2pt}{0pt plus 2pt minus 2pt}
\titlespacing\subsubsection{0pt}{11pt plus 4pt minus 2pt}{0pt plus 2pt minus 2pt}

%------------ Apendices -----------
\usepackage[titletoc,title]{appendix}
\usepackage{tocloft}
% ---------- INCLUIR PDF'S-------------
\usepackage{pdfpages}
%---------------------------------------
\renewcommand\b[1]{\textbf{#1}}
%-------------Numeração romana------------------
\pagenumbering{Roman} % primeiras páginas
 
 %--------- Inicio do documento--------------------
 
\begin{document}

% ----------------------------------------------------
% Cover and Cover Sheet
% ---------------------------------------------------
% !TEX encoding = UTF-8 Unicode

\thispagestyle{empty}

\begin{titlepage}
		\begin{center}
	
		\vspace*{2.5cm}
		\includegraphics[width=0.8\textwidth]{images/isep.pdf}

		\vspace{4.0cm}
	
		\Large{\textbf{Screen missing patients using smart and innovative solutions to eradicate tuberculosis}}\\
			
		\normalsize
		EPCON at Blended-AIM\\
		\large
		 \vspace{0.5cm}
		2019/2020\\
		
		\vspace{1.0cm}
		\normalsize \textbf{1171316 Diogo Capela} \\
			
		\normalsize

		\vspace*{2.8cm}
		\tiny{\includegraphics[width=0.3\textwidth]{images/dei.pdf}}
		
	\end{center}
\end{titlepage}

\cleardoublepage



\thispagestyle{empty}

	\begin{center}
		%	\vspace*{0.5cm}
		

		\vspace*{2.5cm}
		\Large{\textbf{Screen missing patients using smart and innovative solutions to eradicate tuberculosis}}\\
			
		\vspace{0.7cm}
		\normalsize EPCON at Blended-AIM \\
	
		\large
		\vspace*{0.5cm} 2019 / 2020\\
		\vspace*{1.5cm}
		\normalsize \textbf{1171316 Diogo Capela}\\

	
		\vspace*{2.5cm}
	\includegraphics[width=0.55\textwidth]{images/isep.pdf}
	    \vspace*{2.5cm}

		\Large{\textbf{Degree in Software Engineering}}\\
			    \vspace*{1.5cm}
			\normalsize June 2020 \\ \\
				    \vspace*{1.5cm}
			\normalsize ISEP Supervisor: \textbf{Nuno Escudeiro}\\
			\normalsize External Supervisor:  \textbf{Vincent Meurrens}\\

	
\end{center}

%\end{titlepage}

\cleardoublepage

% ----------------------------------------------------------------------
%  Acknowledgment
% ----------------------------------------------------------------------
\chapter*{Acknowledgment}
\vspace{1.2cm}
\input{tex/Acknowledgment}  
\cleardoublepage

% ----------------------------------------------------------------------
%  Abstract
% ----------------------------------------------------------------------
\newpage
\chapter*{Abstract}
\addcontentsline{toc}{chapter}{Abstract}
This report aims to document the project carried out in the 2020 edition of the Blended-AIM program, within the scope of the PESTI curricular unit, in the third year of the degree in software engineering of ISEP.
\\ \\
The main focus of the project was to develop on top of existing AI technology to help EPCON screen missing patients using smart and innovative solutions to eradicate tuberculosis.
\\ \\ \\

\begin{tabular}{>{\raggedright\arraybackslash}p{6.3cm}p{8cm}}
	\textbf{Keywords (Subject):}	
	& Computer Aided Diagnostics, Tuberculosis
	\\ \\
	\textbf{Keywords (Technology):}
	& 	JavaScript, Node.js, Express, PostgreSQL, React, Redux, Android, Java, Kotlin, HTML, CSS \\ 
\end{tabular} 







\cleardoublepage

% ----------------------------------------------------------------------
%  Table of contents, list of tables, list of figures and nomenclature
% ----------------------------------------------------------------------
\newpage
\begingroup

\pagestyle{plain}

\tableofcontents
\cleardoublepage 
 
\newpage
\addcontentsline{toc}{chapter}{Images}
\listoffigures
\cleardoublepage 
 
% --------------------------------------------
% Glossary
% --------------------------------------------
\newpage
\addcontentsline{toc}{chapter}{Glossary}
\chapter*{Glossary}

\paragraph{}
\hspace {0.1cm} 

\begin{tabular}{>{\raggedright\arraybackslash}p{1cm}>{\raggedright\arraybackslash}p{15cm}}

\textbf{AI} &  Artificial intelligence\\
\textbf{CAD} &  Computer aided diagnostics\\
\textbf{CET} &  Central european time\\
\textbf{TB} &  Tuberculosis\\
\textbf{GIS} &  Geographic information system\\
\textbf{NGO} &  Non-governmental organisation\\
\textbf{ML} &  Machine learning\\
\textbf{CD} &  Continuous development\\
\textbf{CI} &  Continuous integration\\

\end{tabular}
\newline

\cleardoublepage 
\endgroup
\pagenumbering{arabic}
\newpage

% -------------------------------------------
% Chapters

\chapter{Introduction}
\label{introduction}
% !TEX encoding = UTF-8 Unicode

\pagenumbering{arabic}

The proposed challenge by EPCON was to develop on top of an already existing system that uses AI technology to help doctors and other healthcare workers screen patients in risk of contracting TB. This project from EPCON was focused on computer-aided diagnosis, specifically for tuberculosis, and for under-developed world regions.

\section{Organization description}

EPCON is a small startup company based in Antwerp, Belgium that since 2018 focuses on identifying, monitoring and evaluating epidemic outbreak regions and finding those in need of help through technology and innovation.
\\ \\
Through their partner network, that consists of a vast list of international companies, NGOs and local communities that have proven track records and unique expertise in the field of GIS, AI and healthcare, they consolidate expertise in terms of health data and commit to further innovate and develop technologies to help epidemic control and fight against TB.
\\ \\
By doing this, they aggregate real-time data and combine it with contextual geographical information like population density, disease co-infection indexes, migration ratios and others. And then, using their distributed Bayesian reasoning models, they analyse cause-effect patterns and use neural networks to screen chest x-rays at large scale for computer-aided diagnostics. \cite{EPCON}

\section{Problem description}

One of the already built technologies of EPCON consists of a small web application that connects to their internal API allowing users to upload a chest x-ray photo and receiving a computer generated estimation on how likely is the x-ray to indicate that the patient could have tuberculosis.
\\ \\
After some testing with real doctors in developing countries, EPCON asked the users for some feedback about the usability of the application and they replied with some issues that concerned their use of it.
\\ \\
One of the biggest complaints was referent to the upload process of the photos of x-rays where the process was taking too long and when the connection was unstable it would fail the request and stop uploading.

\section{Approach}

Bla bla bla

\section{Contributions}

Bla bla bla

\section{Work planning}

We were 10 students from 7 different countries working together remotely for approximately 3 months. We have a first initial meeting during one week in Belgium where we had the chance to meet each other, to get to know the people behind the company and to be elucidated about more specific details on the project we were about to develop.
\\ \\
We decided to have a weekly meeting, every Thursday at 20:00 CET using an online video-conference platform provided to us by the company. Where we discussed all the topics and issues we had during the previous week and all the tasks and user stories we were about to develop on the next week. Vincent, the CEO of EPCON, would join this meeting intercalary, so every two weeks.
\\ \\
In order to communicate with each other we used Slack as it had a free-tier option and it allowed us to communicate, send files and make voice calls.
\\ \\
To manage our workflow and write down all the user stories and their respective status we decided to use Trello, a web-based Kanban-style list-making application. \cite{Trello} We first thought about using Jira, an agile project management software developed by Atlassian, but, as we had some team members who were not very experienced with scrum-based workflow and software development in general, we decided to go with Trello as it seemed a much more simpler, user-friendly option which could fulfill all our organizational needs.
\\ \\
In order to build software in a continuous development fashion we need a version control system where we could track version changes of our code repository. Nowadays, it is almost industry standard to use Git as a version control software, but there are other options like Mercurial or SVN, so we decided to use Git. As a Git platform we used GitLab, as it offered free private repositories and free CI/CD tools.

\section{Report structure}

Apresentação sucinta dos capítulos que fazem parte do relatório, descrevendo em poucos parágrafos o que cada um deles trata. 
Para além da introdução, esta dissertação contém mais x capítulos. No capítulo 2, é descrito o estado da arte e são apresentados trabalhos relacionados. No capítulo 3...





\newpage\cleardoublepage

\chapter{State of the art}
\label{state-of-the-art}
Computer aided detection (CAD) is a clinically proven technology that increases the detection of clinical signs and diseases by assisting the doctors in decreasing observational oversights. The more recent clinical introduction of CAD to assist doctors in the detection of pulmonary signs and diseases will likely be followed by the development, clinical trials validation, regulatory approval and commercialization of a variety of CAD applications in diagnostic imaging. \cite{Castellino2005}
\\ \\
These kinds of software are being used in all fields of medicine but because of the nature of the data itself it is more used in areas where the clinical diagnosis is mostly made by using visual tools, like radiology for example. The computer analyses the image and symptomatic data and outputs a likelihood percentage to aid the doctor in his/her diagnosis.
\\ \\
Nonetheless, making a diagnosis based on images of a patient is often not an easy task. The idea of using computers to help, and possibly improve the interpretation of medical images is therefore very appealing. The first reference to the term computer-aided diagnosis that I have found is in a paper from 1963 by G.S. Lodwick, in Investigative Radiology 1(1):72-80 entitled "Computer-aided diagnosis in radiology. A research plan." \cite{Lodwick1966}

\section{Similar solutions}

There are many companies investing in this field of aided diagnostics, but when analyzing it just for the ones who are focusing on tuberculosis, we can say that EPCON has two main competitors in the market:

\begin{itemize}

\item
\textbf{Qure.ai} from the United States, which has a product named \textbf{qXR} that detects abnormal chest X-rays, then identifies and localizes 15 common abnormalities and screens for tuberculosis. It is used in public health screening programs and was trained with over a million curated X-rays and radiology reports, making it hardware-agnostic and robust to variations in X-ray quality. \cite{QureAI}

\item 
\textbf{Delft Imaging} from the Netherlands, that developed \textbf{CAD4TB} in order to help (non-expert) readers detect tuberculosis more accurately and cost-effectively through the speed of digital X-rays combined with deep learning and remote expertise. \cite{DelftImaging}
	
\end{itemize}

All these technologies have a big focus on computer aided-diagnosis, whoever, we were not developing the CAD AI system ourselves, but rather using the one already built and tested provided by EPCON. Our job was to build an application wrappers around it to make it user friendly and usable for the web and for mobile.

\section{Web application development context}

The way we develop web application has come along way since its beginnings.
\\ \\
During the \textbf{early 90s} was the era of the static HTML pages, where web pages were text documents and only later it was possible to add styles, images, audio and video files.
\\ \\
Around \textbf{1995} the JavaScript language was presented, which made the web pages faster and added the possibility of having dynamic client content and elements on the web pages.
\\ \\
In the year \textbf{1996} Flash was introduced by Macromedia, which allowed to enrich the web pages with interactive animations using a programming language called ActionScript. This allowed for an explosion of interactive web video games on the world wide web.
\\ \\
Around \textbf{2006} there started to be a shift from static to dynamic web applications with the introduction of jQuery, which made DOM manipulation way easier and browser consistent and with the introduction of Ajax technology, which enabled the client (in this case the browser) to make asynchronous requests to REST APIs. During this time it was also introduced the the notion of responsive web design, where developers would build applications that would scale depending on the window width, making it available for smaller screens on mobile and larger screens on desktop, writing the application just once, instead of having two separate versions for mobile and desktop users.
\\ \\
During \textbf{2011} there was a big leap forward for web development because of the introduction of the HTML5 spec. It improved widely the standards, supporting most types of multimedia to be present on the web and allowing to create web applications that are independent from browsers and platforms. The introduction of HTML5 marked the beginning of the decline in the use of Flash for interactive content on web pages. Around this time some web application frameworks started to become popular and widely used, most notably Backbone and Ember.js.
\\ \\
From around \textbf{2016} frameworks like Angular, React and Vue started to emerge and the concept of progressive web application was born. Using the latest browser APIs, like notification, geo-location, and service workers, web applications could behave almost identical to native apps.
\\ \\
Nowadays the development of client-side web applications is dominated by \textbf{React} and \textbf{Vue}, followed by \textbf{Angular}. Concepts like server-side rendering, static HTML generation and posterior hydration with JavaScript and offline usage using service workers have been widely adopted and standardized.

\\ \\
\begin{figure}[H]
	\centering
	\includegraphics[width=1.0\linewidth]{pesti-report/images/web-app-frameworks-graph.jpg}
	\caption{Web application framework popularity by GitHub stars}
	\label{fig:web-app-frameworks-graph}
\end{figure}
\\

\section{Mobile application development context}

The state of the development of mobile applications has been evolving since its conception and it still didn't grew to a completely mature point as progresses are being made still every single day.
\\ \\
In the context of mobile applications we can divide them into two distinct groups: \textbf{native applications} and \textbf{hybrid applications}.
\\ \\
A \textbf{native application} is a software which has been developed to perform some specific task on particular environment or platform, built using a SDK for a certain software framework, hardware platform or operating system. For example, Android applications are build using the Java development kit and Kotlin or Java, iOS applications using the iOS SDK, Swift and Objective-C and Windows applications using the .NET frameworks and the C# language.
\\

\begin{center}
 \begin{tabular}{||c c||}
 \hline
 Operating System & Frameworks and Languages \\ [0.5ex]
 \hline\hline
 Android & JDK, Kotlin, Java  \\
 \hline
 iOS & iOS SDK, Swift, Objective-C  \\
 \hline
 Windows & .NET, C#  \\
 \hline
\end{tabular}
\end{center}

 
\\ \\
An \textbf{hybrid application} has some similarities to the native apps but also some differences. It can be downloaded from the platforms app store just like native apps do, it can get access to most of the native platform features and it's performance can become close to the native app. The major difference is that with an hybrid app the developers can write code just once and publish it in all platforms. The are advantages and disadvantages in deciding to build an hybrid app instead of a native app.
\\ \\
\textbf{Advantages of hybrid apps:}

\begin{itemize}
\item
Typically faster to develop
\item 
Unified development of having a single code base
\item 
Less expensive to build and maintain
\item
Ability to target a wider user pool without much effort

\end{itemize}

\\ \\
\textbf{Disadvantages of hybrid apps:}

\begin{itemize}

\item
Hybrid UI feels less responsive to users
\item 
Indirect access to device hardware and software APIs, usually accessed through framework wrappers
\item 
Worse performance

\end{itemize}

As of today, there are three popular frameworks for creating hybrid mobile applications: \textbf{Ionic}, \textbf{React Native} and \textbf{Flutter}.
\\ \\

\begin{figure}[H]
	\centering
	\includegraphics[width=1.0\linewidth]{pesti-report/images/mobile-hybrid-frameworks-graph.jpg}
	\caption{Mobile hybrid framework popularity by GitHub stars}
	\label{fig:mobile-hybrid-frameworks-graph}
\end{figure}


\textbf{Ionic} is a framework that allows developing applications for iOS and Android. It embeds the application inside a browser without the navigation options (or WebView) that can be run as a standalone application on any platform. \cite{IonicGitHub}
\\ \\
\textbf{React Native} uses a very different approach to browser-based described above. Instead of rendering content in a WebView or the mobile web browser, the content is rendered with native OEM components provided by the platform. JavaScript is still used for the application logic to be able to re-use the same logic across platforms, and the rendering, which is the most performance-sensitive part, is done with native components for increased performance. \cite{ReactNativeGitHub}
\\ \\
\textbf{Flutter} is the latest and more trending nowadays. It was developed by Google and uses a programming language named Dart, which is also maintained by Google. It provides very good and fast tools to work, build and run the code and it maintains solid consistency between all platforms. \cite{FlutterGitHub}
\newpage\cleardoublepage

\chapter{Solution analysis and design} \label{cap:solution-analysis-and-design}
EPCON is a startup company that works in multiple fields of digital social or medical software like epidemics and NGO support. This project from EPCON was focused on computer-aided diagnosis, specifically for tuberculosis, and for under-developed world regions.

\section{Problem domain}

We were told about the problem and all the details when we first met with EPCON in Belgium. During this meeting they showed us their API and a demo web application they had built that connected to this API. It allowed users to upload a pulmonary X-ray image and to get an output image with the areas where signs were found and a likelihood of tuberculosis. \cite{Biometrics}
\\
See the screenshot of the app below:
\\ \\

\begin{figure}[!h]
	\centering
	\includegraphics[width=0.8\linewidth]{pesti-report/images/epcon-web-app.jpg}
	\caption{EPCON demo web application}
	\label{fig:epcon-web-app}
\end{figure}

\\

This demo application had three main problems:

\begin{itemize}
\item
It connected directly with the artificial intelligence API, which made all the requests very slow
\item
It didn't use the latest symptoms payload they have added to the API
\item
It didn't persist data
\end{itemize}

So, our project was to build a system that would consume this machine learning API they have created that addressed all the listed problems and more.
\\ \\
EPCON wanted the app to be free to use for anyone, but they also wanted to have a registration system for doctors who wanted to persist all their screenings and patients. The app should allow doctors to manage patients and screen their respective X-rays.

\section{Domain model}

The first step was to identify the different domains we would be working with. This was a time-consuming process because all of us were completely new to this computer-aided diagnosis area and also because all of us came from different countries and were speaking a non-native language. Nevertheless, we came up with these three different domains:

\begin{itemize}
\item
User
\item
Patient
\item
Screening
\end{itemize}

A \textbf{user} is anyone who is using the app. The app is targeted to be use by doctors, healthcare professionals or medical students, but it is free and open to anyone, so we thought it would be better to abstract from the concept of doctor.
\\ \\
A \textbf{patient} is a record added and persisted by a user and can be referenced on a screening.
\\ \\
A \textbf{screening} is an attempt to diagnose an X-ray pulmonary image.


\\ \\
\begin{figure}[!h]
	\centering
	\includegraphics[width=1.0\linewidth]{pesti-report/images/domain-model.jpg}
	\caption{Domain model}
	\label{fig:domain-model}
\end{figure}
\\


\section{Functional requirements}

One of our first approaches during the first meeting was to try to define user-stories and write all of them down.

\\ \\
\begin{figure}[!h]
	\centering
	\includegraphics[width=1.0\linewidth]{pesti-report/images/use-case-diagram.jpg}
	\caption{Use case diagram}
	\label{fig:use-case-diagram}
\end{figure}

\\
Here are the use cases we collected:

\begin{itemize}
\item
A user can register using email, password and phone number
\item
A user can login using email and password
\item
A logged user can logout
\item
A logged user can view his/her profile
\item
A logged user can edit his/her profile (name, city, country and bio)
\item
A logged user can edit his/her password
\item
A logged user can edit his/her phone number
\item
A logged user can edit his/her email address
\item
A logged user can delete his/her account
\item
A user can submit a screening for diagnosis
\item
A user can view the diagnosis of a specific screening
\item
A user can view the list of his/her screenings
\item
A user can review a diagnosed screening
\item
A user can delete a specific screening
\item
A logged user can add a patient
\item
A logged user can view the list of his/her patients
\item
A logged user can view a specific patient details
\item
A logged user can edit a patient details(name, sex and year of birth)
\item
A logged user can delete a specific patient
\item
A logged user can reference a patient when submitting a screening
\item
A user can read the version changelog
\item
A user can read the about statement
\item
A user can read the terms and conditions
\item
A user can read the privacy policy
\item
A user can read the cookie policy
\end{itemize}
\newpage\cleardoublepage

\chapter{Solution development}
\label{cap:solution-development}
% !TEX encoding = UTF-8 Unicode

EPCON had an already built API which provided computer-aided diagnostics. This API accepted an x-ray photo encoded using base 64 and returned a diagnosis estimation. This is a computer-intensive process and it usually took between 10 to 40 seconds to respond. The point was to create a backend RESTful API to hide complexity and manage those heavy requests and then develop clients that would connect to this facade API.
\\ \\
This new API would also deliver features that were not present on the EPCON API, like user authentication, patient and screening data storage.

\\ \\
\begin{figure}[!h]
	\centering
	\includegraphics[width=1.0\linewidth]{pesti-report/images/global-system.jpg}
	\caption{System architecture}
	\label{fig:global-system}
\end{figure}
\\

The objective was to build clients with a focus on user-experience and which felt faster than directly requesting the EPCON servers.

\section{Implementation description}

To decide which technologies and frameworks we would use in order to develop the whole system we started by gathering information about the technical skills and previous experiences each of us had. We strongly thought about it and tried to not be influenced by the framework hype that exists nowadays. Many times during softwares development people put too much emphasis on the technology rather than the solution. We all agreed that we would go with the technologies that we felt comfortable with and not the ones that seemed better.
\\ \\
After this information gathering we decided to go with Node.js and Express for the backend API, JavaScript and React for the web client and Kotlin and Java for the Android client.

\\ \\
\begin{figure}[!h]
	\centering
	\includegraphics[width=1.0\linewidth]{pesti-report/images/database-schema.jpg}
	\caption{Database schema}
	\label{fig:database-schema}
\end{figure}
\\


\section{Backend architecture}


\\ \\
\begin{figure}[!h]
	\centering
	\includegraphics[width=1.0\linewidth]{pesti-report/images/backend-server-architecture.jpg}
	\caption{Backend server architecture}
	\label{fig:backend-server-architecture}
\end{figure}
\\


\section{Web client structure}

\\ \\
\begin{figure}[!h]
	\centering
	\includegraphics[width=1.0\linewidth]{pesti-report/images/web-client-structure.jpg}
	\caption{Web client structure}
	\label{fig:web-client-structure}
\end{figure}
\\


\section{Android client structure}


\\ \\
\begin{figure}[!h]
	\centering
	\includegraphics[width=1.0\linewidth]{pesti-report/images/android-client-structure.jpg}
	\caption{Android client structure}
	\label{fig:android-client-structure}
\end{figure}
\\




\section{Tests}


\\ \\
\begin{figure}[!h]
	\centering
	\includegraphics[width=1.0\linewidth]{pesti-report/images/unit-tests.jpg}
	\caption{Unit tests}
	\label{fig:unit-tests}
\end{figure}
\\

\section{Releases and deployment}

\section{Solution evaluation}


\newpage\cleardoublepage

\chapter{Conclusion}
\section{Objectives achieved}

The objectives we set during our first meeting were mainly this three:

\begin{itemize}

\item
Build an RESTful API wrapper around EPCON's API with user authentication and persistent data
\item 
Build a web client with focus on user experience and performance
\item 
Build a mobile client with focus on user experience and performance

\end{itemize}

Our project was ambitious, we were 11 students from different countries and communicating in English (which for most of us, including myself, was not our native language), some more experienced than others in terms of working remotely, agile methodology and software development.
\\ \\
We successfully developed the REST API with persistent data and an authentication system and the two clients (web and Android). As of this moment, there are still bugs and issues to work on and the Android app was not yet released on the Google Play Store. Nevertheless, I think we mostly achieved the objectives we have defined at the beginning of the project.

\section{Limitations and future work}

During the execution and development of this project we have faced multiple challenges, some more with more impact than others. Here are some issues we have faced:

\begin{itemize}

\item
Cultural differences between team members
\item 
Working remotely on different time-zones
\item 
Experience gap in terms of agile software development

\end{itemize}

Most of the difficulties we faced were related to team-work issues. All of us were working remotely, in different countries, some in a very different time-zone which created barriers of communication. Another issue we faced was very different ranges of technical knowledge in software development tools and processes. Some of our team members were not familiar with popular programming languages like Java or JavaScript, others had not yet been introduced to versioning tools like Git and most were not familiar with the processes that software development takes, like writing user-stories, following a backlog and working following agile methods.
\\ \\
We have recently published all our code with a permissive open source license on GitHub. So, as of today anyone in the world can start contributing for the project. We plan on continuing to develop the system until September, where our main focus will be on finishing the Android app and bug fixing some issues on the server and on the web client.

\section{Final assessment}

Having the opportunity to work on this project was a pleasure for me. It strengthened my knowledge both in a technical perspective and a team-work related view. I've learned a lot about APIs and the Android SDK and working with such a culturally diverse team made me rethink on how much communication is important.
\\ \\
I feel that EPCON was appreciative of our work and recognized our efforts and in the end we effectively build a tool that was asked for us. Overall it was an amazing experience that I think everyone will take some knowledge home.
\newpage\cleardoublepage

% -------------------------------------------
% Bibliography

\renewcommand\bibname{Bibliography}
\addcontentsline{toc}{chapter}{Bibliography}
\bibliographystyle{apalike} 
\bibliography{bibliography/references}   



% ------------------------------------------
% Anexos

\begin{appendix}
\titleformat{\chapter}{\bfseries\huge}{\appendixname{} \thechapter.}{20pt}{\bfseries\huge}

\renewcommand{\appendixname}{Attachment}
\addtocontents{toc}{\protect\contentsline {chapter}{Attachments}{}{}}

\addtocontents{toc}{\cftpagenumberson{chapter}} 
\renewcommand{\appendixname}{Attachment}

\chapter{Attachment A Title}
\label{attachment-a}
\input{pesti-report/tex/attachments/attachment-a}
\newpage
\cleardoublepage

\chapter{Attachment B Title}
\label{attachment-b}
% !TEX encoding = UTF-8 Unicode

Esta parte do relatório deve conter informação adicional organizada por capítulos que, embora seja interessante, não faz parte do material estritamente necessário ao relatório. Documentos importantes produzidos ou utilizados durante o estágio que, pela sua dimensão, não sejam colocáveis no corpo principal do relatório podem ser incluídos em anexos.

\newpage
\cleardoublepage

\chapter{Attachment C Title}
\label{attachment-c}
\input{pesti-report/tex/attachments/attachment-c}
\newpage
\cleardoublepage

\end{appendix}
\end{document}

